\documentclass[a4paper,12pt]{report}
\usepackage[utf8]{inputenc}
\usepackage[ngerman]{babel}
\usepackage{titlesec}
\usepackage{graphicx}
\usepackage{setspace}
\usepackage{lipsum}
\usepackage{geometry}
\usepackage{tocbibind}
\usepackage{hyperref}
\usepackage{array}

\geometry{a4paper, left=25mm, right=25mm, top=30mm, bottom=30mm}

\titleformat{\chapter}{\normalfont\huge}{\thechapter.}{20pt}{\huge\bfseries}
\titleformat{\section}{\normalfont\Large}{\thesection}{12pt}{\Large\bfseries}
\titleformat{\subsection}{\normalfont\large}{\thesubsection}{8pt}{\large\bfseries}

\title{Webbasierte Multiplayer Schach-App}
\author{Jasper Paul Fülle}
\date{24. Mai 2023}

\begin{document}

\begin{titlepage}
    \begin{center}
        \vspace*{1cm}
        
        \Large
        \textbf{Webbasierte Multiplayer Schach-App}
        
        \vspace{0.5cm}
        
        Bachelorarbeit
        
        \vspace{1.5cm}
        
        \normalsize
        vorgelegt von\\
        Jasper Paul Fülle\\
        Matrikelnummer 3367654
        
        \vspace{1cm}
        
        Betreut von\\
        Prof. Dr. Thorsten Thormälen
        
        \vspace{1cm}
        
        \begin{tabular}{>{\raggedright}p{4cm}>{\raggedleft}p{4cm}}
        Studiengang & Wirtschaftsinformatik \\
        \end{tabular}
        
        \vspace{1cm}
        
        \today
        
        \vfill
        
        
        \vspace{0.5cm}
        
        Fachbereich Mathematik und Informatik\\
        Philipps Universität Marburg
    \end{center}
\end{titlepage}

\tableofcontents

\chapter{Einleitung}
    \section{Motivation}
    
    \section{Zielsetzung}
    \section{Aufbau der Arbeit}

\chapter{Theoretische Grundlagen}
    \section{Schach}
    \section{Web-Technologien}
        \subsection{Node.js und Express}
        \subsection{Socket.io}
        \subsection{React}
        \subsection{PostgreSQL}

\chapter{Systemarchitektur}
    \section{Frontend}
        \subsection{React-Komponenten}
        \subsection{Benutzerführung}
        \subsection{Kommunikation mit dem Backend}
    \section{Backend}
        \subsection{API-Endpunkte}
        \subsection{Datenbankstruktur}
        \subsection{Echtzeit-Kommunikation}

\chapter{Implementierung}
    \section{Frontend-Entwicklung}
    \section{Backend-Entwicklung}
    \section{Datenbank-Integration}

\chapter{Tests und Evaluation}
    \section{Funktionalitätstests}
    \section{Usability-Tests}
    \section{Performancetests}
    \section{Sicherheitstests}

\chapter{Fazit und Ausblick}
    \section{Zusammenfassung der Ergebnisse}
    \section{Limitationen}
    \section{Potenzielle Erweiterungen und Weiterentwicklung}

\appendix
\chapter{Anhang}
    \section{Installationsanleitung}
    \section{Quellcode}
    \section{Abbildungsverzeichnis}

\listoffigures
\listoftables

\bibliographystyle{unsrt}
\bibliography{bibliography}

\end{document}

