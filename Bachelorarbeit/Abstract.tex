
\begin{abstract}
%Viele der in der Computergrafik verwendeten 3D-Modelle werden mit Hilfe von Dreiecksnetzen repräsentiert. ... (max. 1 Seite)

Das Interesse an Schach hat in den letzten Jahren immer mehr zugenommen und Online-Schachplattformen verzeichnen aktuell Rekorde an Benutzern und täglichen Spielen. Dies bietet die attraktive Möglichkeit eine Schach-App zu entwerfen und zu implementieren, deren Konzept Elemente aufgreift,

 welche bei bisherigen Schachplattformen kritisiert werden (kann ich eigentlich nicht schreiben, weil das nur meine Meinung ist ;()

welche mehr Anreize schaffen Partien zu spielen und soziale Interaktionen mit Fremden aber auch innerhalb einer Freundesgruppe zu fördern. 

Diese Bachelorarbeit hat das Ziel eine Webbasierte Multiplayer Schach-App zu entwerfen, die die Basis für Erweiterungen bildet, um konkurrenzfähig gegenüber den bisherigen Schachplattformen zu sein. Der Fokus liegt dabei besonders auf soziale Interaktionen, eine ansprechende Benutzererfahrung und Sicherheit hinsichtlich Benutzerdaten.
\end{abstract}
 
\newpage\thispagestyle{empty}\hspace{1em}\newpage

\begin{otherlanguage}{english}
\begin{abstract}
text text text text text text
text text text text
text text text text text text text text text text
(exakte englische Übersetzung der deutschen Kurzzusammenfassung)
\end{abstract}
\end{otherlanguage} 
