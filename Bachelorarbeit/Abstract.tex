
\begin{abstract}
%Viele der in der Computergrafik verwendeten 3D-Modelle werden mit Hilfe von Dreiecksnetzen repräsentiert. ... (max. 1 Seite)

Das Interesse an Schach hat in den letzten Jahren immer mehr zugenommen und Online-Schachplattformen verzeichnen aktuell Rekorde an Benutzern und täglichen Spielen. Vor diesem Hintergrund zielt diese Bachelorarbeit darauf ab eine Webbasierte Multiplayer Schach-App zu entwerfen die den Anforderungen von Schachspielenden im digitalen Zeitalter gerecht wird und gleichzeitig als Basis für zukünftige Erweiterungen dient.

Die entwickelte Schach-App bietet Funktionen wie das Spielen von mehreren Schachpartien mit unterschiedlichen Zeitkonfigurationen und einem Chat gleichzeitig, das Registrieren und Anmelden auf der Plattform, das Hinzufügen und Herausfordern zu Schachpartien befreundeter Personen, sowie das Zuschauen eines Spiels. Dabei wird Wert auf moderne Webtechnologien gelegt, und der Entwurf der Architektur ermöglicht eine gute Erweiterbarkeit, Skalierbarkeit, Wartbarkeit und Modularität.

Im Frontend wird die React-Bibliothek mit Chakra-UI verwendet, um eine optimale Benutzererfahrung zu schaffen. Das Backend basiert auf Node.js mit dem Express-Framework und nutzt eine PostgreSQL- und eine Redis-Datenbank, um die Speicherung der Daten möglichst effizient zu gestalten. Die Kommunikation erfolgt über die Node.js-API und Socket.io, welche eine bidirektionale Echtzeitkommunikation zwischen den Clients und dem Server ermöglicht.

Die Ergebnisse dieser Arbeit zeigen, dass die entwickelte Schach-App den gestellten Anforderungen gerecht wird und ein solides Fundament für zukünftige Erweiterungen und Verbesserungen bietet. Die gewonnenen Erkenntnisse können in weiteren Projekten zur Entwicklung des Backends und Frontends von Webanwendungen, insbesondere mit Echtzeitkommunikation, genutzt werden.
\end{abstract}
 
\newpage\thispagestyle{empty}\hspace{1em}\newpage

\begin{otherlanguage}{english}
\begin{abstract}
text text text text text text
text text text text
text text text text text text text text text text
(exakte englische Übersetzung der deutschen Kurzzusammenfassung)
\end{abstract}
\end{otherlanguage} 
